\section{Lecture 6: Bayesian Updating }


\subsection*{Learning from Data Example}
A disease has prevalence \(P(H^+) = 0.005\).  
A screening test has 2\% false positives and 1\% false negatives.

We observe a \textbf{positive test result} \(T^+\).

We want \(P(H^+|T^+)\) and \(P(H^-|T^+)\).

\subsection*{Bayes’ Theorem}
\(
P(H|D)=\frac{P(D|H)\,P(H)}{P(D)} \\
P(D)=\sum_i P(D|H_i)P(H_i)
\)

% \subsection*{Tree Representation}
% \(
% \begin{array}{rcl}
% P(T^+|H^+) &=& 0.99,\\
% P(T^-|H^+) &=& 0.01,\\
% P(T^+|H^-) &=& 0.02,\\
% P(T^-|H^-) &=& 0.98.
% \end{array}
% \) \\
Total probability: \\
\(
P(T^+)=0.99(0.005)+0.02(0.995)\\=0.02485.
\)

\subsection*{Posterior Probabilities}
\(
P(H^+|T^+)=\frac{0.99\cdot0.005}{0.02485}=0.199,\quad \\
P(H^-|T^+)=0.801.
\) \\
A positive test increases the probability of disease but it remains below 50\%.

\subsection*{Terminology}
Data \(D\): observed evidence (here \(T^+\)). \\
Hypotheses \(H\): \(H^+\) (has disease), \(H^-\). \\
Likelihoods \(P(D|H)\): probability of data given hypothesis. \\
Priors \(P(H)\): belief before seeing data. \\
Posteriors \(P(H|D)\): belief after observing data.



\subsection*{Likelihood Table}
\(
\begin{array}{lcc}
\text{Hypothesis} & P(T^+|H) & P(T^-|H) \\
\hline
H^+ & 0.99 & 0.01\\
H^- & 0.02 & 0.98
\end{array}
\)

\subsection*{Bayesian Update Table}
\resizebox{\linewidth}{!}{%
\(
\begin{array}{lcccc}
H & P(H) & P(T^+|H) & P(T^+|H)P(H) & P(H|T^+)\\
\hline
H^+ & 0.005 & 0.99 & 0.00495 & 0.199\\
H^- & 0.995 & 0.02 & 0.01990 & 0.801\\
\text{Total} & 1 &  & 0.02485 & 1
\end{array}
\)
}




\subsection*{Law of Total Probability}
\(
P(D_1)=\sum_i P(D_1|H_i)P(H_i) \\
P(D_2|D_1)=\sum_i P(D_2|H_i)P(H_i|D_1).
\)

\subsection*{Probabilistic Prediction}
After observing \(D_1=5\),\\
\(
P(D_2=4|D_1=5) \\ =\sum_i P(D_2=4|H_i)P(H_i|D_1)=0.124.
\)\\
This allows computing the probability of future outcomes given past data.

\subsection*{Iterated Updates in Odds Form}
For independent data \(D_1,D_2\):
\(
O(H|D_1,D_2)=O(H)\cdot\mathrm{BF}_1\cdot\mathrm{BF}_2,
\)
where the Bayes factor \\ \(\mathrm{BF}_i=P(D_i|H)/P(D_i|H^c)\).

